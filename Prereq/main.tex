\documentclass{article}
\usepackage[utf8]{inputenc}
\usepackage{amsmath}
\usepackage{amssymb}
\usepackage[nottoc]{tocbibind}
\usepackage{parskip}
\usepackage{amsthm}
\usepackage{hyperref}

\title{CMSC351: Prerequisite}
\author{Johning To}
\date{5/23/2023}

\renewcommand{\contentsname}{}

\begin{document}
\maketitle

\tableofcontents
\clearpage

\section{Symbols}
\begin{enumerate}
  \item For all ($ \forall $)
  \item There exists ($ \exists $)
\end{enumerate}

\clearpage

\section{Proofs}
\subsection {Weak Induction}
First, we need to prove $ \forall n \ge n_0 $ $ P(n) $ we first prove $ P(n_0) $ (which is the base case) and
then we prove $ \forall k \ge n_0 \; P(k) \rightarrow P(k+1) $ (which is the inductive step) (you can also prove $ P(k - 1) $).
The assumption of $ P(k) $ in the inductive step is the inductive hypothesis.

Ultimately, for the inductive step we are trying to find that for any $ k \ge n_0 $ if $ P(k) $ is true, then $ P(k + 1) $ is also true. 

\textbf{Example:}
Suppose we have a set of nested Russian dolls (Matryoshka dolls). Each doll is contained with another doll, each doll is labeled 1, 2, 3 and so on. 

In this hypothetical, there are two things that are true. Let's say $ M(n) $ is true iff (if and only if) doll $ n $ has another doll contained.

(a) The first doll has another doll contained. That is, $ M(n) $ is true.

(b) For every doll k, if doll k has another doll inside then doll $ k + 1 $ has another doll inside.
$$ \forall k \ge 1, M(k) \rightarrow M(k + 1) $$

We can now conclude that $ \forall n \ge 1, M(n) $.

\subsection {Strong Induction}
The goal of strong induction is we need to find some property, we can say $ P(n) $ and 
we need to find some n greater than equal to a ($ n \ge a $).

How can we accomplish this?

\textbf {Step 1 (Basis step):} We are going to prove for $ P(a) $ (we would just prove $ P(a) $ for weak induction), $ P(a + 1) $, ...,
for some finite number say $ P(b) $. 

($P(a),\; P(a + b),\;...,\;P(b)$) $ \leftarrow $ we prove each of these.

\textbf {Step 2 (Induction):} We can now assume $ P(i) $ where $ a \le i \le k $. (Assume $ P(i) $).
Then, we prove $ P(k + 1) $.

\textbf {Example: Fibonacci Sequence}
\begin{proof}
Claim: The Fibonacci Sequence is defined as the following: $ F(0) = 0, F(1) = 1 $ and for $ n \ge 2, F(n) = F(n - 1) + F(n - 2). $
We want to prove that for all $ n \ge 0 , F(n) \le 2^n $

Base cases:
\begin{alignat*}{2}
For\; n = 0: F(0) = 0 \le 2^0 = 1
\\
For\; n = 1: F(1) = 1 \le 2^1 = 1
\end{alignat*}
For both these base case, they hold true. 

Inductive Step: 
Assume that for all $ i $, $ 0 \le i \le k $, we have $ F(k) \le 2 ^ k $.

(This could also be known as the inductive hypothesis)

We now need to prove $ F(k + 1) \le 2 ^ {k + 1} $.

By the inductive hypothesis, we have $ F(n) \le 2^n $ and $ F(n - 1) \le 2^{n-1} $.

We can prove by which:
\begin{center}
  $\begin{aligned}
    F(k+1) & = F(k) + F(k-1)
    \\
    & \le 2^{k-1} + 2^{k-2} & \text{(IH)}
    \\
    & \le 2^{k-1} + 2^{k-1} & \text{($ 2^{k-2} $ is less than $ 2^{k-1} $)}
    \\
    & = 2^k & \text{(Simplify)} 
  \end{aligned}$
\end{center}
\end{proof}

\subsection{Constructive Induction}
When we are solving recurrences and we have guessed the general form, and we do not know the constants, we usepackage
constructive induction.

\textbf{Example:}
We know that $ \sum_{i=1}^{n} i = \frac{1}{2}n^2 - \frac{1}{2}n$. But how do we determine $ \sum_{i=1}^{n} i^2 $?
Since we know the solution for the first sum is a quadratic, we can guess that for the second sum that is cubic 
($an^3 + bn^2 + cn + d$).
We start with assumptions. $\sum_{i=1}^{n-1} i = a(n-1)^3 + b(n-1)^2 + c(n-1) + d$ and $ n > 0 $
We need to prove that $\sum i = 1^ni^2 = an^3 + bn^2 + cn + d$
So, we need:

\begin{alignat*}{2}
  \sum_{i=1}^{n} i = an^3 + bn^2 + cn + d \\
  \sum_{i=1}^{n-1} i + n^2 = an^3 + bn^2 + cn + d \\
  a(n-1)^3 + b(n-1)^2 + c(n-1) + d + n^2 = an^3 + bn^2 + cn + d \\
  a(n^3-3n^2+3n-1) + b(n^2-2n+1) + c(n-1) + d = an^3 + bn^2 + cn + d \\
  an^3 + (b-3a)n^2 + (3a-2b+c)n + (d-a+b-c) = an^3 + bn^2 + cn + d \\  
  an^3 + (b-3a+1)n^2 + (3a-2b+c)n+(d-a+b-c) = an^3 + bn^2 + cn + d
\end{alignat*}
We arrive at a systems of equations
\begin{alignat*}{2}
  b-3a+1 = b \\
  3a-2b+c = c \\
  d-a+b-d = d \\
\end{alignat*}
Thus, $ a = \frac{1}{3} $ , $ b=\frac{1}{2} $ , $ c=\frac{1}{6} $ and $ d = 0 $

Therefore, 
\begin{alignat*}{1}
  \sum_{i=1}^{n} i^2 = \frac{1}{3}n^3 + \frac{1}{2}n^2 + \frac{1}{6}n = \frac{n(n+1)(2n+1)}{6}
\end{alignat*}

\subsection{Structural Induction}
\end{document}