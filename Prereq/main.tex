\documentclass{article}
\usepackage[utf8]{inputenc}
\usepackage{amsmath}
\usepackage{amssymb}
\usepackage[nottoc]{tocbibind}
\usepackage{parskip}
\usepackage{amsthm}
\usepackage{hyperref}

\title{CMSC351: Prerequisite}
\author{Johning To}
\date{5/23/2023}

\renewcommand{\contentsname}{}

\begin{document}
\maketitle

\tableofcontents
\clearpage

\section{Symbols}
\begin{enumerate}
  \item For all ($ \forall $)
  \item There exists ($ \exists $)
\end{enumerate}

\clearpage

\section{Proofs}
\subsection {Weak Induction}
First, we need to prove $ \forall n \ge n_0 $ $ P(n) $ we first prove $ P(n_0) $ (which is the base case) and
then we prove $ \forall k \ge n_0 \; P(k) \rightarrow P(k+1) $ (which is the inductive step) (you can also prove $ P(k - 1) $).
The assumption of $ P(k) $ in the inductive step is the inductive hypothesis.

Ultimately, for the inductive step we are trying to find that for any $ k \ge n_0 $ if $ P(k) $ is true, then $ P(k + 1) $ is also true. 

\textbf{Example:}
Suppose we have a set of nested Russian dolls (Matryoshka dolls). Each doll is contained with another doll, each doll is labeled 1, 2, 3 and so on. 

In this hypothetical, there are two things that are true. Let's say $ M(n) $ is true iff (if and only if) doll $ n $ has another doll contained.

(a) The first doll has another doll contained. That is, $ M(n) $ is true.

(b) For every doll k, if doll k has another doll inside then doll $ k + 1 $ has another doll inside.
$$ \forall k \ge 1, M(k) \rightarrow M(k + 1) $$

We can now conclude that $ \forall n \ge 1, M(n) $.

\subsection {Strong Induction}
The goal of strong induction is we need to find some property, we can say $ P(n) $ and 
we need to find some n greater than equal to a ($ n \ge a $).

How can we accomplish this?

\textbf {Step 1 (Basis step):} We are going to prove for $ P(a) $ (we would just prove $ P(a) $ for weak induction), $ P(a + 1) $, ...,
for some finite number say $ P(b) $. 

($P(a),\; P(a + b),\;...,\;P(b)$) $ \leftarrow $ we prove each of these.

\textbf {Step 2 (Induction):} We can now assume $ P(i) $ where $ a \le i \le k $. (Assume $ P(i) $).
Then, we prove $ P(k + 1) $.

\textbf {Example: Fibonacci Sequence}
\begin{proof}
Claim: 
The Fibonacci Sequence is defined as the following: $ F(0) = 0, F(1) = 1 $ and for $ n \ge 2, F(n) = F(n - 1) + F(n - 2). $
We want to prove that for all $ n \ge 0 , F(n) \le 2^n $

Base cases:

For $ n = 0: F(0) = 0 \le 2^0 = 1 $

For $ n = 1: F(1) = 1 \le 2^1 = 1 $

For both these base case, they hold true. 

Inductive Step: 
Assume that for all $ i $, $ 0 \le i \le k $, we have $ F(k) \le 2 ^ k $.

(This could also be known as the inductive hypothesis)

We now need to prove $ F(k + 1) \le 2 ^ {k + 1} $.

By the inductive hypothesis, we have $ F(n) \le 2^n $ and $ F(n - 1) \le 2^{n-1} $.

We can prove by which:
\begin{center}
  $\begin{aligned}
    F(k+1) & = F(k) + F(k-1)
    \\
    & \le 2^{k-1} + 2^{k-2} & \text{(IH)}
    \\
    & \le 2^{k-1} + 2^{k-1} & \text{($ 2^{k-2} $ is less than $ 2^{k-1} $)}
    \\
    & = 2^k & \text{(Simplify)} 
  \end{aligned}$
\end{center}
\end{proof}

\end{document}