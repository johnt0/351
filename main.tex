\documentclass{article}
\usepackage[utf8]{inputenc}
\usepackage{pgfplots}
\pgfplotsset{compat=1.8, width=10cm}
\usepackage{amsmath}
\usepackage{amssymb}


\title{CMSC351: Big Notation}
\author{Johning To}
\date{5/22/2023}

\begin{document}
\maketitle

\section{The Big-O Notation}

\textbf {Definition:}
For Big O, we say
$$ f(x) = O(g(x)) \text { if } \exists x_0, C > 0 \; \text {such that} \; \forall x \ge x_0, f(x) \le Cg(x) $$
Essentially, we need to find some $ C $ where $ g(x) $ is greater than $ f(x) $ at some $ x_0 $.
Example: 

\text {Notice that }
\\ \\
\textbf {Definition:}
For Big $ \Omega $, we say
$$ f(x) = \Omega(g(x)) \text { if } \exists x_0, B > 0 \; \text {such that} \; \forall x \ge x_0, f(x) \ge B(g(x)) $$

\end{document}